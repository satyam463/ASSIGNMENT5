\documentclass[journal,12pt,twocolumn]{IEEEtran}

\usepackage{setspace}
\usepackage{gensymb}

\singlespacing


\usepackage[cmex10]{amsmath}
%\usepackage{amsthm}
%\interdisplaylinepenalty=2500
%\savesymbol{iint}
%\usepackage{txfonts}
%\restoresymbol{TXF}{iint}
%\usepackage{wasysym}
\usepackage{amsthm}
%\usepackage{iithtlc}
\usepackage{mathrsfs}
\usepackage{txfonts}
\usepackage{stfloats}
\usepackage{bm}
\usepackage{cite}
\usepackage{cases}
\usepackage{subfig}
%\usepackage{xtab}
\usepackage{longtable}
\usepackage{multirow}
%\usepackage{algorithm}
%\usepackage{algpseudocode}
\usepackage{enumitem}
\usepackage{mathtools}
\usepackage{graphicx}
\usepackage{refstyle}
\usepackage{caption}
\usepackage{steinmetz}
\usepackage{tikz}
%\usepackage{circuitikz}
\usepackage{verbatim}
\usepackage{tfrupee}
\usepackage[breaklinks=true]{hyperref}
%\usepackage{stmaryrd}
\usepackage{tkz-euclide} % loads  TikZ and tkz-base
%\usetkzobj{all}
\usetikzlibrary{calc,math}
\usepackage{listings}
   \usepackage{color}                                            %%
    \usepackage{array}                                            %%
    \usepackage{longtable}                                        %%
    \usepackage{calc}                                             %%
    \usepackage{multirow}                                         %%
    \usepackage{hhline}                                           %%
    \usepackage{ifthen}                                           %%
  %optionally (for landscape tables embedded in another document): %%
    \usepackage{lscape}     
%\usepackage{multicol}
\usepackage{chngcntr}
%\usepackage{enumerate}

%\usepackage{wasysym}
%\newcounter{MYtempeqncnt}
\DeclareMathOperator*{\Res}{Res}
%\renewcommand{\baselinestretch}{2}
\renewcommand\thesection{\arabic{section}}
\renewcommand\thesubsection{\thesection.\arabic{subsection}}
\renewcommand\thesubsubsection{\thesubsection.\arabic{subsubsection}}

\renewcommand\thesectiondis{\arabic{section}}
\renewcommand\thesubsectiondis{\thesectiondis.\arabic{subsection}}
\renewcommand\thesubsubsectiondis{\thesubsectiondis.\arabic{subsubsection}}

% correct bad hyphenation here
\hyphenation{op-tical net-works semi-conduc-tor}
\def\inputGnumericTable{}                                 %%

\lstset{
%language=C,
frame=single, 
breaklines=true,
columns=fullflexible
}

\begin{document}

\newtheorem{theorem}{Theorem}[section]
\newtheorem{problem}{Problem}
\newtheorem{proposition}{Proposition}[section]
\newtheorem{lemma}{Lemma}[section]
\newtheorem{corollary}[theorem]{Corollary}
\newtheorem{example}{Example}[section]
\newtheorem{definition}[problem]{Definition}

\newcommand{\BEQA}{\begin{eqnarray}}
\newcommand{\EEQA}{\end{eqnarray}}
\newcommand{\define}{\stackrel{\triangle}{=}}
\bibliographystyle{IEEEtran}
%\bibliographystyle{ieeetr}
\providecommand{\mbf}{\mathbf}
\providecommand{\pr}[1]{\ensuremath{\Pr\left(#1\right)}}
\providecommand{\qfunc}[1]{\ensuremath{Q\left(#1\right)}}
\providecommand{\sbrak}[1]{\ensuremath{{}\left[#1\right]}}
\providecommand{\lsbrak}[1]{\ensuremath{{}\left[#1\right.}}
\providecommand{\rsbrak}[1]{\ensuremath{{}\left.#1\right]}}
\providecommand{\brak}[1]{\ensuremath{\left(#1\right)}}
\providecommand{\lbrak}[1]{\ensuremath{\left(#1\right.}}
\providecommand{\rbrak}[1]{\ensuremath{\left.#1\right)}}
\providecommand{\cbrak}[1]{\ensuremath{\left\{#1\right\}}}
\providecommand{\lcbrak}[1]{\ensuremath{\left\{#1\right.}}
\providecommand{\rcbrak}[1]{\ensuremath{\left.#1\right\}}}
\theoremstyle{remark}
\newtheorem{rem}{Remark}
\newcommand{\sgn}{\mathop{\mathrm{sgn}}}
%\providecommand{\abs}[1]{\left\vert#1\right\vert}
\providecommand{\res}[1]{\Res\displaylimits_{#1}} 
%\providecommand{\norm}[1]{\left\lVert#1\right\rVert}
\providecommand{\norm}[1]{\lVert#1\rVert}
\providecommand{\mtx}[1]{\mathbf{#1}}
%\providecommand{\mean}[1]{E\left[ #1 \right]}
\providecommand{\fourier}{\overset{\mathcal{F}}{ \rightleftharpoons}}
%\providecommand{\hilbert}{\overset{\mathcal{H}}{ \rightleftharpoons}}
\providecommand{\system}{\overset{\mathcal{H}}{ \longleftrightarrow}}
	%\newcommand{\solution}[2]{\textbf{Solution:}{#1}}
\newcommand{\solution}{\noindent \textbf{Solution: }}
\newcommand{\cosec}{\,\text{cosec}\,}
\providecommand{\dec}[2]{\ensuremath{\overset{#1}{\underset{#2}{\gtrless}}}}
\newcommand{\myvec}[1]{\ensuremath{\begin{pmatrix}#1\end{pmatrix}}}
\newcommand{\mydet}[1]{\ensuremath{\begin{vmatrix}#1\end{vmatrix}}}
%\numberwithin{equation}{section}
\numberwithin{equation}{subsection}
%\numberwithin{problem}{section}
%\numberwithin{definition}{section}
\makeatletter
\@addtoreset{figure}{problem}
\makeatother
\let\StandardTheFigure\thefigure
\let\vec\mathbf
%\renewcommand{\thefigure}{\theproblem.\arabic{figure}}
\renewcommand{\thefigure}{\theproblem}
%\setlist[enumerate,1]{before=\renewcommand\theequation{\theenumi.\arabic{equation}}
%\counterwithin{equation}{enumi}
%\renewcommand{\theequation}{\arabic{subsection}.\arabic{equation}}
\def\putbox#1#2#3{\makebox[0in][l]{\makebox[#1][l]{}\raisebox{\baselineskip}[0in][0in]{\raisebox{#2}[0in][0in]{#3}}}}
     \def\rightbox#1{\makebox[0in][r]{#1}}
     \def\centbox#1{\makebox[0in]{#1}}
     \def\topbox#1{\raisebox{-\baselineskip}[0in][0in]{#1}}
     \def\midbox#1{\raisebox{-0.5\baselineskip}[0in][0in]{#1}}
\vspace{3cm}
\title{Assignment-5}
\author{Satyam Singh \\ EE20MTECH14015}
\maketitle
\newpage
\bigskip
\renewcommand{\thefigure}{\theenumi}
\renewcommand{\thetable}{\theenumi}
\begin{abstract}
This assignment deals with Q R decomposition .
\end{abstract}
Download all python codes from 
\begin{lstlisting}
https://github.com/satyam463/ASSIGNMENT5/blob/master/code.py
\end{lstlisting}
\section{Problem Statement}
 Perform QR decomposition on matrix $\vec{A}$
\begin{align}
    \vec{A}=\myvec{ 1 & -7 \\ 3 & 1 }\label{givmat}
\end{align}
\section{Solution}
The columns of matrix $\vec{A}$ can be represented in $\alpha$ and $\beta$ as
\begin{align}
    \implies\alpha=\myvec{1 \\ 3}\\
    \beta=\myvec{-7\\ 1}
\end{align}
For QR decomposition, matrix $\vec{A}$ can be expressed as
\begin{align}
    \vec{A}=\vec{Q}\vec{R}\label{decomp}
\end{align}
where, $\vec{Q}$ and $\vec{R}$ are expressed as
\begin{align}
    \vec{Q}=\myvec{\vec{u_1} & \vec{u_2}}\label{Q}\\
    \vec{R}=\myvec{k_1 & r_1 \\ 0 & k_2}\label{R}
\end{align}
Note that $\vec{R}$ is an upper triangular matrix.\\
Now,we calculate
\begin{align}
    k_1=\norm{\alpha}=\sqrt{10}\\
    \vec{u_1}=\frac{\alpha}{k_1}=\frac{1}{\sqrt{10}}\myvec{1 \\ 3}\\
    r_1=\frac{\vec{u_1}^T\beta}{\norm{\vec{u_1}}^2}=\frac{1}{\sqrt{10}}\myvec{1 & 3}\myvec{-7 \\ 1}\\
    \implies r_1=-\frac{4}{\sqrt{10}}\\
    \vec{u_2}=\frac{\beta-r_1\vec{u_1}}{\norm{\beta-r_1\vec{u_1}}}\label{u2}
\end{align}
Consider
\begin{align}
    \beta-r_1\vec{u_1}=\myvec{-7 \\ 1}+\frac{4}{\sqrt{10}}\frac{1}{\sqrt{10}}\myvec{1 \\ 3}\\
    \implies\beta-r_1\vec{u_1}=\myvec{\frac{-66}{10}\\\frac{22}{10}}\label{num}\\
    \norm{\beta-r_1\vec{u_1}}=\frac{{22}}{\sqrt{10}}\label{den}
\end{align}
Substitute \refeq{num},\refeq{den} in \refeq{u2}, we get
\begin{align}
    \vec{u_2}=\myvec{-\frac{3}{\sqrt{10}} \\ \frac{1}{\sqrt{10}}}\\
    k_2=\vec{u_2}^T\beta=\myvec{-\frac{3}{\sqrt{10}} & \frac{1}{\sqrt{10}}}\myvec{-7 \\ 1}\\
    \implies k_2=\frac{22}{\sqrt{10}}
\end{align}
Therefore, from \refeq{Q} and \refeq{R}
\begin{align}
    \vec{Q}=\myvec{\frac{1}{\sqrt{10}} & -\frac{3}{\sqrt{10}} \\ \frac{3}{\sqrt{10}} & \frac{1}{\sqrt{10}}}\\
    \vec{R}=\myvec{\sqrt{10} & -\frac{4}{\sqrt{10}} \\ 0 & \frac{22}{\sqrt{10}}}
\end{align}
Note that,
\begin{align}
    \vec{Q}^T\vec{Q}=\myvec{\frac{1}{\sqrt{10}} & \frac{3}{\sqrt{10}} \\ -\frac{3}{\sqrt{10}} & \frac{1}{\sqrt{10}}}\myvec{\frac{1}{\sqrt{10}} & -\frac{3}{\sqrt{10}} \\ \frac{3}{\sqrt{10}} & \frac{1}{\sqrt{10}}}=\myvec{1 & 0\\ 0 & 1}=\vec{I}
\end{align}
Now matrix $\vec{A}$ can be written as \refeq{decomp}
\begin{align}
    \myvec{1 &-7 \\ 3 & 1}=\myvec{\frac{1}{\sqrt{10}} & -\frac{3}{\sqrt{10}} \\ \frac{3}{\sqrt{10}} & \frac{1}{\sqrt{10}}}\myvec{\sqrt{10} & -\frac{4}{\sqrt{10}} \\ 0 & \frac{22}{\sqrt{10}}}
\end{align}
\end{document}
